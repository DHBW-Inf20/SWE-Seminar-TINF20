\subsection{Preisvergleich}
Jeder Anbieter bietet verschiedene Preis- und Abrechnungsmodelle an und stellt verschiedene KI-Lösungen zur Verfügung. Ein eindeutiger Preisvergleich über alle Funktionen zwischen den verschiedenen Unternehmen ist daher nur begrenzt möglich. Um trotzdem einen direkten Vergleich zu erhalten, haben wir uns die Speech-to-Text-Lösungen der einzelnen Anbieter angeschaut. Diese wurde gewählt, da sie von allen Firmen angeboten wird und es sich hierbei um eine Standardlösung handelt, die in vielen Bereichen angewendet werden kann und kein „Nischenprodukt“, das speziell für eine Branche entwickelt wurde, ist. Bei Speech-to-Text wird gesprochene Sprache erkannt und in geschriebenen Text umgewandelt. Die Anwendungsgebiete dafür reichen von intelligenten persönlichen Assistenten wie Apples Siri oder Amazons Alexa, über das Mitschreiben und Dokumentieren von Meetings oder Interviews, bis hin zur barrierefreien Gestaltung von Softwareprodukten, um Menschen mit Behinderung einen leichteren Zugriff auf digitale Inhalte zu ermöglichen. Zusätzlich gibt es zum Teil Zusatzfunktionen, die das Gesprochene gleichzeitig noch übersetzen oder sogar den Sprechenden verifizieren oder identifizieren können. \\

Google bietet seinen Kunden eine kostenlose Nutzung der Speech-to-Text-API von 60 Audiominuten pro Monat. Für alles darüber hinaus berechnet Google pro 15 Sekunden mindestens 0,004 \$. Das sind 0,016 \$ pro Minute. Dabei wird jede Anfrage auf die nächsten vollen 15 Sekunden aufgerundet. Bestimmt wird der endgültige Preis durch weitere Faktoren wie die Spracherkennung mit einem erweiterten KI-Modell, dem aktivieren von Daten-Logging oder der Anzahl der erkannten Audiokanäle der Daten. Begrenzt ist die Nutzung der API auf 1.000.000 Audiominuten pro Monat. Möchte man mehr Sprache erkennen, muss man bei Google eine Kontingentanfrage einreichen und Google über seinen Bedarf aufklären. \cite[vgl.][]{GoogleCloud.PV.2022} \\ 
Amazon Transcribe bietet seinen Kunden ebenfalls 60 Freiminuten pro Monat, allerdings nur für ein Jahr. Spätestens danach muss man auch hier nach Audiominuten dafür bezahlen. Hierbei ist das Preismodell Stufenweise mit drei verschiedenen Tier-Stufen aufgebaut. Für die ersten 250.000 Minuten im Monat zahlt man 0,024 \$/Minute. Für die nächsten 750.000 0,015 \$/Minute und für jede weitere Minute über 1.000.000 0,0102 \$. \cite[vgl.][]{AWS.PV.2022} \\
Bei den Microsoft Speech Services erhalten Kunden in der kostenlosen Version pro Monat fünf Audiostunden umsonst. Um mehr Daten zu verarbeiten, bietet Microsoft diesen Service für einen Dollar pro Audiostunde an. Das entspricht etwa 0,0167 \$/Minute. Außerdem bietet Microsoft auch verschiedene „Commitment Tiers“ an. Darauf muss man sich bewerben und kann dann für einen fixen Preis eine fixe Anzahl an Audiostunden erwerben. So zahlt man für 2.000 Stunden 1.600 Dollar, für 10.000 Stunden 6.500 Dollar und für 50.000 Stunden 25.000 Dollar. \cite[vgl.][]{Azure.PV.2022} \\
Auch IBM mit dem Watson Speech-to-Text-Service bietet verschiedene Preispläne an. Mit dem „Lite“-Plan erhält man kostenlos pro Monat 500 Audiominuten. Mit dem „Plus“-Plan zahlt man bis zu 1.000.000 Minuten 0,02 Dollar pro Minute, alles darüber kostet dann nur noch die Hälfte, nämlich 0,01 Dollar pro Minute. Zusätzlich gibt es noch die Pläne „Premium“ und „Deploy Anywhere“. Diese bieten weitere Zusatzfunktionen, wie zum Beispiel eine zusätzliche Datenverschlüsselung. Die Preise dieser Pläne sind allerdings nur auf Nachfrage einsehbar. \cite[vgl.][]{IBM.PV.2021} \\

Die Tabelle (siehe \autoref{tab:preisvergleich}) fasst die Preise der Anbieter zusammen. \\

\begin{table}[]
\def\arraystretch{2}
\begin{tabular}{|l|l|l|l|}
\hline
          & kostenlos                        & Weiterführend                                                                                                                     & Begrenzung       \\ \hline
Google    & 60min/M                       & 0.016\$/min                                                                                                                      & 1.000.000min/M \\ \hline
Amazon    & 60min/M (1 Jahr lang)                & \begin{tabular}[c]{@{}l@{}}0.024\$/min(bis 250.000min)\\ 0,015\$/min(bis 1.000.000min)\\ 0.0102\$/min(danach)\end{tabular}          & keine            \\ \hline
Microsoft & 300min/M (kostenlose Version) & \begin{tabular}[c]{@{}l@{}}\textbf{Einfacher Gebrauch:} 0,0167\$/min\\ \textbf{Commitment Tiers:}\\ 1.600\$ für 2.000h\\ 6.500\$ für 10.000h\\ 25.000\$ für 50.000h\end{tabular} & keine            \\ \hline
IBM       & Lite-plan 500min/M             & \begin{tabular}[c]{@{}l@{}}0.02\$/min (für 1.000.000min)\\ 0.01\$/min (danach)\end{tabular}                                      & keine            \\ \hline
\end{tabular}
\caption{Preisvergleich}\label{tab:preisvergleich}
\end{table}