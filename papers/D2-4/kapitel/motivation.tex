\subsection{Motivation}
Unsere Welt befindet sich in einem andauernden Umbruch. Immer mehr befindet sich in einem stetigen Wandel. Alles wird schneller und kurzlebiger. So ist es auch in der Informatik, speziell in der Softwareentwicklung bzw. der Bereitstellung von Software. Die Anforderungen an ein heutiges System sind vor allem Flexibilität und Agilität. Hier kommt die Idee von XaaS ins Spiel. Unter XaaS versteht das Prinzip von Everything-as-a-Service. Unter diesem Ansatz versteht man hauptsächlich Dienstleistungen rund um Infrastruktur (IaaS), Software (SaaS) und Plattformen (PaaS) als einen Service zu beziehen. Erweitert reicht XaaS aber auch bis hin zum Einsetzen von künstlicher Intelligenz als Service (AIaaS). Bei einer solchen Umstellung auf serviceorientierte Betriebsmodelle kann der Verwaltungsaufwand erheblich sinken, da man sich nicht mehr um die Wartung und Instandhaltung der genutzten Dienste kümmern muss. Außerdem können durch ein solches Modell die Anwendungen und Systeme nach Bedarf skaliert werden. Dadurch entfallen die Kosten für die Altsysteme und es entstehen ausschließlich neue Kosten für die in Anspruch genommenen Leistungen. \\