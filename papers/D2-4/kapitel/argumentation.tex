\section{Vor- und Nachteile von AIaaS}
\subsection{Vorteile}
AIaaS zu nutzen kann viele Vor- aber auch Nachteile mit sich bringen. \\ \\
Wird AIaaS in einem Unternehmen genutzt, können Risiken verhindert werden, die in Verbindung mit dem Entwickeln und Warten von eigenen KI-Infrastrukturen auftreten könnten. Beim Neuentwickeln eigener KI´s ist keine Sicherheit gegeben, dass diese von Anfang an wie gewünscht funktioniert. Die Entwicklung von KI´s kann sehr komplex werden und kann deshalb bei Unternehmen, die bisher eher geringe Berührungspunkten mit KI´s hatten zu Fehleinschätzungen führen. Ein weiteres Risiko wäre zum Beispiel der Kontrollverlust der KI. Wird unbedacht oder blind auf diese Systeme vertraut, kann die Unabhängigkeit und Entscheidungsfreiheit verloren werden. Mit Nutzung von AIaaS können Risiken wie diese zwar nicht ausgelöscht werden, jedoch ist dort ein Know-how vorhanden, welches man selbst möglicherweise nicht besitzt. \\
Ein bekannter Fall, bei dem sich eine Künstliche Intelligenz selbstständig machte, ist ein Forschungsprojekt aus dem Jahr 2016. Hier versuchten zwei Forscher aus dem Google Brain Team zwei Maschinen miteinander kommunizieren zu lassen. Die Aufgabe dabei war, das Die Maschinen namens Bob und Alice Nachrichten verschicken, die nur der Gegenüber entziffern kann. Dagegen arbeitete eine dritte Maschine Namens Eve, die versuchte die Nachrichten zu entziffern. Anfangs konnte Eve viele Nachrichten entziffern, jedoch lernte Bob schnell und nach etwa 15 000 Durchläufen hatte Eve keine Chancen mehr die Nachrichten immer korrekt zu entziffern. Bob und Alice hatten einen Verschlüsselungsalgorithmus gefunden, den nur die beiden entschlüsseln konnten. Selbst die Wissenschaftler des Google-Teams konnten die Nachrichten nicht mehr nachvollziehen, da sie vergessen hatten zu implementieren, das die Bots sich ausschließlich auf Englisch unterhalten sollten und mussten das Projekt vorerst abbrechen. \cite[vgl.][]{Zeit.2016} \cite[vgl.][]{giga.2017} \\ \\
%https://www.zeit.de/digital/datenschutz/2016-10/google-kuenstliche-intelligenz-erfindet-eigene-verschluesselung?utm_referrer=https%3A%2F%2Fwww.google.com%2F    
% https://www.wissenschaftsjahr.de/2019/das-wissenschaftsjahr/da-kommt-ja-kein-mensch-drauf/was-reden-die-denn-da-chatbots-geben-raetsel-auf/index.html
Außerdem kann durch das Nutzen von AIaaS Diensten den Unternehmen bei der Einhaltung von Datenschutz- und Sicherheitsvorschriften helfen, da diese mit den Regelungen und Vorschriften vertraut sein sollten. Falls personenbezogene Daten vom Dienstleister verarbeitet werden, muss der Nutzer einen Auftragsverarbeitungs-Vertrag schließen. Der abgeschlossene AV-Vertrag regelt die Rechte und Pflichten des Auftraggebers und Auftragnehmers sowie etwaiger eingesetzter Sub-Dienstleister. Es sollte sichergestellt werden, dass  Auftragnehmer die einem anvertrauten Daten nur für den Zweck verarbeitet werden, für den der Auftraggeber die Daten erhoben hat. Vor allem aber sind Dienstleister verpflichtet,  Daten in angemessenem Umfang zu schützen. Um dies in der Praxis sicherzustellen, gibt der Vertrag dem Kunden diesbezüglich umfassende Kontrollrechte. \cite[vgl.][]{ActiveMindAG.2022} \\ \\
%https://www.activemind.de/datenschutz/dokumente/av-vertrag/
Mit AIaaS können KI-Fähigkeiten schnell und effizient skaliert werden. Das bedeutet, dass bei hohem Gebrauch der Dienste, mehr Cloud-Speicher dazu gebucht werden kann. Das gleiche gilt auch für die andere Richtung. Werden weniger Kapazitäten benötigt, so kann die Nutzung herunter skaliert werden. Testet ein Unternehmen beispielsweise mehrere Anbieter, um zu sehen, welche am bestem geeignet für es ist, so werden geringere Kapazitäten benötigt, wie bei einer Vollauslastung auf ein anstehendes Projekt bei einem Anbieter. \cite[vgl.][]{Cuofano.2022} \\
Die Lufthansa Industry Solutions beschreiben die Skalierbarkeit ihrer Service mit folgendem Satz:\glqq AIaaS kann je nach Bedarf und Wachstum des Unternehmens nach oben oder unten skaliert werden. Sämtliche KI-Tools lassen sich einzeln hinzufügen, sodass Unternehmen auch mit kleinen Schritten den Weg zur Nutzung von künstlicher Intelligenz gehen können\grqq. \cite[vgl.][]{LIS.2022} \\ \\
%https://www.lufthansa-industry-solutions.com/de-de/loesungen-produkte/kuenstliche-intelligenz/ai-as-a-service-automatisierung-von-geschaeftsprozessen
Ein weiterer Vorteil für die Nutzung von AIaaS ist der Zugang zu den neuesten und fortschrittlichsten KI-Technologien für die Unternehmen. Anbieter von AIaaS müssen stets auf dem aktuellen Stand sein, um auf dem Markt kompetitiv zu bleiben. Dadurch kann als Verbraucher immer eine verlässliche und fortschrittliche Technik erworben werden. Im Vergleich zu einer eigenen Implementierung wird auch hier viel Zeit gespart. Es würde einige Zeit dauern, um qualitativ auf das Level der Anbieter in dem Gebiet zu kommen. \cite[vgl.][]{Folio.2022} \\ \\
Der wohl größte Vorteil, der sich durch AIaaS ergibt, sind die Kosteneinsparungen. Durch Nutzung einer KI-Dritter, werden die kompletten Kosten für Entwicklung und Wartung einer eigenen KI eingespart. Es müssen ebenfalls keine Mitarbeiterkosten dafür gezahlt werden. Ein weiterer Vorteil, der sich dadurch ergibt, ist dass man sich auf seine Kernkompetenzen konzentrieren und die Pflege der KI-Infrastruktur den Experten überlassen kann.\\
Zahlen der Lufthansa Industry Solutions ergeben, dass die Effektivität der Prozesse um 60\% gesteigert werden können und die damit einhergehende Zeitersparnis eine Kostensenkung von bis zu 20 Prozent zur Folge hat. Des weiteren wird ein Tool zur Verfügung gestellt, bei dem man seine Anfragen pro Monat, die Minuten zur Bearbeitung einer Anfrage und die Kosten pro Stunde angibt. Mit diesen Informationen lassen sich die aktuellen Kosten in etwa bestimmen und gleichzeitig lassen sich die Kosten bei Nutzung von AIaaS ansehen.
%https://www.lufthansa-industry-solutions.com/de-de/loesungen-produkte/kuenstliche-intelligenz/ai-as-a-service-automatisierung-von-geschaeftsprozessen 
%gleich wie oben valla
Somit kann sich der Kunde ein Bild davon machen, ob es sich lohnt, AIaas einzuführen oder nicht. \cite[vgl.][]{LIS.2022} \\ 

\subsection{Nachteile}
Es gibt demzufolge viele Vorteile, die AIaaS mit sich bringt. Jedoch gibt es auch einige Nachteile, weshalb AIaaS nicht die perfekte Lösung darsellen könnte:\\
Es besteht immer die Gefahr, dass KI als Dienstleistung zur Entwicklung von autonomen Waffen oder anderen gefährlichen Technologien genutzt werden könnte. Auch wenn der Fall eher unwahrscheinlich ist, muss dieser beachtet werden. Der Nutzer muss nicht einmal unbedingt etwas böses im Sinn haben. Es reicht aus, wenn durch Fehler die künstliche Intelligenz intelligenter als der Mensch wird. Dadurch wird die KI unkontrollierbar und unvorhersehbar und es entstehen potenzielle Bedrohungen für die Welt und unsere Spezies. \\
In der 42. Deutschen Konferenz über Künstliche Intelligenz ging es unter anderem um KI-Systeme für das Militär. KI-Systeme werden sowohl für autonome Waffensysteme, als auch zur Lagebeurteilung benutzt. Damit soll frühzeitig erkannt werden, ob beispielsweise ein feindlicher Raketenangriff droht. Somit kann ein menschlicher Entscheidungsträger schnell Vergeltungsschläge ausführen. Laut Experten könnten solche KI's zur Destabilisierung der internationalen Beziehungen führen und auch zu Konflikten, die sich verselbstständigen. Viele Länder möchten ein Verbot autonomer Waffen. Jedoch lehnen große Waffenhersteller wie Russland, die USA und Israel die Forderungen ab. \cite[vgl.][]{Deutschlandfunk.2019}

%https://www.deutschlandfunk.de/autonome-waffen-ki-systeme-im-militaer-100.html
Der Einsatz von AIaaS könnte zu Arbeitsplatzverlusten und anderen wirtschaftlichen Störungen führen. Logischerweise folgt aus den Kosteneinsparungen bei Nutzung von KI´s ein geringeres Personal, das benötigt wird. Setzen sich Dienstleistungen und KI´s in einem größerem Feld durch, können viele Arbeiter ihren Arbeitsplatz dadurch verlieren. Das kann wiederum zu wirtschaftlichen Störungen auf dem Markt führen.

Ein anderer Punkt ist, dass AIaas das Risiko von Cyberangriffen und anderen bösartigen Aktivitäten erhöhen könnte. Durch die einfache Möglichkeit, an qualitativ hochwertige KI´s zu kommen, können Menschen mit boshaften Absichten diese KI´s ausnutzen, um Leute oder gar Firmen mit Cyberangriffen zu drohen oder gar durchzuführen.

Letzten Endes muss noch angesprochen werden, dass AIaaS genutzt werden kann, um Nutzer auszubeuten und zu manipulieren sowie ihre Privatsphäre zu verletzen. KI´s sind sehr modern und viel gefragt. Dadurch kann ein potenzieller Nutzer sehr einfach manipuliert werden, sehr hohe Preise für eine kleine KI zu zahlen. Außerdem kann die Privatsphäre von Kunden sehr schnell angegriffen werden. Zum Lernen benötigt die KI meist sehr viele Daten. Oft werden dafür auch Daten von Kunden genommen, die normalerweise einen privaten Status haben. \cite[vgl.][]{InApp.2020}

Ein aktueller Fall zeigt, dass das Einsetzen und zur Verfügung stellen von einem KI-Tool auch rechtliche Konsequenzen für den Anbieter haben kann, welcher die Schäden aber auch an seine Kunden weitergeben könnte. Eine aktuelle Sammelklage richtet sich gegen die Unternehmen GitHub, den Mutterkonzern Microsoft und das KI-Unternehmen OpenAI. Die von GitHub angeboten Programmierhilfe mit dem Namen GitHub Copilot soll laut Klage keine Quellen angeben, woher der generierte Code kommt, der zur Verfügung gestellt wird. Dadurch werden die Open-Source-Lizenzmodelle und die Rechte der Programmierer des Codes verletzt. \\
Copilot ist seit Juni 2022 zugänglich und bietet dem Nutzer Vorschläge, wie der momentane Code vervollständigt werden kann. Das KI-System greift dabei auf viele öffentliche Code-Repositorys in GitHub zurück. Dabei wird der Ersteller es Codes jedoch nicht um Erlaubnis gefragt. Die Klage beschreibt einen Verstoß gegen die Richtlinien des Digital Millennium Copyright Act (DMCA). Laut Klage verstößt die Software drei Mal gegen Abschnitt 1202 DMCA – fehlender Verweis aufs Urheberrecht, fehlende Namensnennung, sowie fehlender Lizenztext. Würde man davon ausgehen, dass jeder Copilot Nutzer einmal gegen den Abschnitt 1202 verstoßen hat, so würden sich 3.600.000 Verstöße durch GitHub ansammeln. Der Mindestschadenersatz läge dabei bei 2500 US-Dollar pro Einzelfall. Somit kommt man auf eine Gesamtsumme von 9 Milliarden Dollar. Dazu kommen noch weitere in der Klageschrift vorgeworfene Rechtsverstöße wie unlauterer Wettbewerb oder der Verstoß gegen das kalifornische Verbraucherschutzgesetz. \cite[vgl.][]{Wittenhorst.2022}  \\