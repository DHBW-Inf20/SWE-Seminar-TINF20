\newpage
\section{Artificial Intelligence-as-a-Service (AIaaS)}
Ein stark wachsender Bereich in der Digitalisierung und der Automatisierung ist die Künstliche Intelligenz. Sie kann für eine Vielzahl von Aufgaben eingesetzt werden. Sie kann zum Beispiel Analysen und Erkenntnisse liefern, Maschinen helfen Menschen zu verstehen und mit ihnen zu interagieren oder alltägliche Aufgaben automatisieren. KI wird auch im Gesundheits-, Bank- und Finanzwesen eingesetzt, um Prozesse zu automatisieren, Prognosemodelle zu erstellen und die Kundenerfahrung zu verbessern. \\
Das Erstellen und Trainieren von KI-Modellen kann allerdings sehr zeitaufwendig und gegebenenfalls auch sehr komplex sein. Außerdem benötigen KI-Anwendungen in der Regel eine hohe Rechenleistung. Somit ist es häufig in kleinen bis mittleren Unternehmen nicht gegeben, dass zum einen das notwendige KI-KnowHow und KI-Entwickler vorhanden sind und zum anderen auch die benötigte Infrastruktur nicht gegeben ist. Aus diesen Gründen gibt es auch hier eine Nachfrage nach einem cloudbasierten Produkt, mit welchem sich, ohne großes Vorwissen und Aufwand, künstliche Intelligenzen in die eigenen Anwendungen einbinden lassen. Die Lösung dafür nennt sich Artificial Intelligence-as-a-Service, oder kurz AIaaS. \cite[vgl.][441f.]{Lins.2021} \\ \\
Artificial Intelligence-as-a-Service (AIaaS), oder Künstliche Intelligenz als Service, ist ein Cloud-basierter Dienst, der es Unternehmen ermöglicht, Funktionen einer künstlichen Intelligenz in ihre Anwendungen und Systeme zu integrieren, ohne die erforderliche Infrastruktur aufbauen oder warten zu müssen. AIaaS-Lösungen können eine Reihe von Funktionen bieten, von der Verarbeitung natürlicher Sprache, Bilderkennung und maschinellem Lernen bis hin zu prädiktiven Analysen und der Automatisierung von Robotikprozessen. Sie können Unternehmen in die Lage versetzen, ihre betriebliche Effizienz zu steigern, Kosten zu senken und den Kundenservice zu verbessern.

Für den Zugriff auf die AIaaS-Lösung zahlen Unternehmen in der Regel eine Abonnementgebühr an den Anbieter, und sie können dann über eine API oder eine andere Integrationsmethode auf die KI-Funktionen zugreifen. Diese Art von Service kann Unternehmen die Ressourcen zur Verfügung stellen, die sie benötigen, um die Entwicklung von KI-Anwendungen zu beschleunigen, ohne in kostspielige Infrastruktur und Technologie investieren zu müssen. \cite[vgl.]{Cuofano.2022}