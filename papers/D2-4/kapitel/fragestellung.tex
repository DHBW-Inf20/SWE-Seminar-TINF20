\subsection{Fragestellung}
Immer mehr Anwendungen basieren auf einer Künstlichen Intelligenz. Ob es diverse Bilderkennungstools, Chatbots oder maschinelle Übersetzer sind, immer mehr setzt man auf die neuronalen Netze.
Bei einer genaueren Betrachtung von XaaS ergeben sich mehrere Fragen, welche es zu beantworten gilt. Zum einen, was ist Everything-as-a-Service? Darunter zählt was man unter XaaS versteht, sowie in welche Unterpunkte man den Begriff untergliedern kann. Der Fokus soll dabei auf AIaaS, also Artificial Intelligence-as-a-Service (Künstliche Intelligenz als Service), liegen. Hierbei wird auf den Nutzen, die möglichen Anwendungsbereiche, sowie die Vor- und Nachteile eingegangen, um abwägen zu können, welche Faktoren AIaaS bzw. XaaS allgemein begünstigen und weshalb sie von immer mehr Unternehmen eingesetzt werden. Außerdem soll analysiert werden, auf welche Serviceanbieter hier in der Regel zurückgegriffen wird, welche Funktionen und Tools sie anbieten und wie dabei das jeweilige Preismodell aufgebaut ist. \\