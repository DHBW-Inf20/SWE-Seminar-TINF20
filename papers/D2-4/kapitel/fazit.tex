\newpage
\section{Fazit}
\subsection{Argumentationsfazit}
Zusammenfassend lässt sich also sagen, dass es durchaus Zahlreiche Argumente gibt, die für den Einsatz von AIaaS im Unternehmensumfeld sprechen. Allerdings gibt es auch einige Gegenargumente die potenziell auftretende Probleme und Gefahren beschreiben. Hierbei gilt, dass jedes Unternehmen für sich erörtern und entscheiden muss wie mit diesen Punkten umgegangen wird und ob man die eventuellen Sicherheitslücken mit eigenen Ressourcen und Know-How schließen kann. \\
Man kann hier also deutlich sehen, dass, auch wenn das so von den verschiedenen Service Providern vermarktet und angepriesen wird, es nicht möglich, bzw. nicht empfehlenswert ist, einfach eine vorhandene künstliche Intelligenz in seine Systeme einzubinden. Man muss zahlreiche Modifikationen vornehmen. Allerdings: Hat man einmal initial die erforderlichen Anpassungen vorgenommen, hat man in der Regel weniger Wartungsaufwand als bei einer hauseigenen on-premise-Lösung, da das weitere Trainieren der KI und auch weitere Optimierungen bezüglich der Sicherheit oder Performance von den Providern durchgeführt wird. \\

\subsection{Anbieterfazit}
Wenn man sich für einen AIaaS-Dienst entschieden hat, geht es bei der Anbieterwahl primär darum, welcher von ihnen eine Lösung anbietet, die für mein Produkt geeignet ist. Viel Provider bieten eine Vielzahl an Diensten an, welche sich nur wenig unterscheiden. Neben den oben genannten Hauptakteuren am Markt kann man sich natürlich auch für einen kleineren Anbieter entscheiden. Das bietet den Vorteil, dass man eventuell einen größeren Verhandlungsspielraum hat, wenn es um Anfragen zu Zusatzfunktionen oder Anpassungen des KI-Modells geht. Bei großen Firmen wie Google oder Amazon hat man bei solchen Anfragen meist keinen Erfolg, da die Produkte für die breite Masse konzipiert sind und Individualanpassungen eher unüblich sind. Was jedoch ein großer Vorteil bei den ,,Big Playern`` ist, ist dass sie schon ein globales Netz an Rechenzentren zur Verfügung haben und somit eine hohe Ausfallsicherheit gewährleistet ist. Außerdem ist bei diesen Anbietern garantiert, dass die KI-Services kontinuierlich gewartet, erweitert und verbessert werden, da es dedizierte Teams dafür gibt.

\newpage
\subsection{Zukunftsaussicht}
Die Zukunft von Everything-as-a-Service sieht sehr vielversprechend aus. Getrieben wird das Wachstum von XaaS durch die Kombination aus der hohen Nachfrage an Cloud Computing und der wachsenden Infrastruktur des globalem Internets mit einer hohen Bandbreite. Im Zuge diesen technologischen Fortschritts nutzen Unternehmen vor allem immer mehr Software-as-a-Service-Lösungen (SaaS) für ihren Betrieb. Es ist zu erwarten, dass sich in Zukunft immer mehr Unternehmen für dieses Modell entscheiden werden, das es ihnen ermöglicht, schnell und einfach auf die von ihnen benötigten Dienste zuzugreifen, ohne in teure Hardware und Software investieren zu müssen. Da Cloud Computing immer beliebter wird, werden auch die Kosten für XaaS-Lösungen immer attraktiver für Unternehmen aller Größenordnungen.

Der globale XaaS-Markt wurde 2021 von,,Allied Market Research`` auf etwa 474,9 Milliarden US-Dollar bemessen. Voraussichtlich wird der Markt 2031 auf bis zu 2,63 Billionen US-Dollar anwachsen. Das wäre eine Durchschnittliche jährliche Wachstumsrate von 18,9 Prozent. Dabei gibt es vor allem starke Nachfragen bei SaaS, aber auch bei IaaS und PaaS. Hauptsächliche Treiber des Wachstums sind hierbei Regierungen, die Gesundheitsindustrie und die produzierende Industrie.

Der Artificial-Intelligence-as-a-Service-Markt für sich wird zwar im vergleich zum allgemeinen XaaS-Markt voraussichtlich prozentual deutlich stärker anwachsen, wird jedoch trotzdem nur einen kleinen Teil des Gesamtmarktes ausmachen. Die Nachfrage an den klassischen Hauptkomponenten ist aktuell noch sehr hoch und und die Anwendungsgebiete für AIaaS, vor allem in kleinen bis mittleren Unternehmen, noch nicht sehr ausgereift. Viele Unternehmen werden sich in Zukunft wohl zunächst auf die reine Digitalisierung bzw. der digitalen Abbildung von Geschäftsprozessen fokusieren, um dann danach Ansätze zu Prozessoptimierung mittels künstlicher Intelligenz zu verfolgen. \cite[vgl.][]{AlliedMarketResearch.2022} \\