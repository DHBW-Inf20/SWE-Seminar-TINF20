\section{Schluss}
\label{sec:schluss}
Abschließend soll in einer kurzen Zusammenfassung die Arbeit zusammengefasst und die Erkenntnisse zentral gesammelt werden. Danach folgt mit den durch die Arbeit erlangten Erkenntnisse ein Ausblick in die Zukunft von intelligenten Werkzeugen zur Softwareentwicklung. 

In der vorliegenden Arbeit wurden wichtige intelligente Werkzeuge zur Softwareentwicklung aufgezeigt. Dazu wurden die theoretischen Grundlagen der einzelnen Werkzeuge genauer aufgezeigt und die in einer Marktanalyse ausgewählte Werkzeuge, welche in der Praxis Anwendung finden, näher analysiert. Abschließend wurden der Nutzen und die möglichen Risiken der einzelnen Werkzeuge evaluiert. Die ausgewählten Werkzeuge, welche vorgestellt wurden, erleichtern den Entwicklungsprozess in essenzieller Weise und gestalten die Arbeit effizienter. Genauer wurden intelligente Werkzeuge zur Codevervollständigung, Codegenerierung, Analyse, Refactoring, Dokumentation und Kollaboration vorgestellt. 
Bei der Marktanalyse ist aufgefallen, dass es eine Vielzahl von Anbietern intelligenter Werkzeuge gibt. Zwar gibt es teilweise sprachübergreifende Werkzeuge, dies ist allerdings nicht immer der Fall. Manchmal kann daher eine gesonderte Analyse bezüglich verwendetem Technologie-Stack sinnvoller sein. 
Des Weiteren ist ersichtlich, dass bestimmte Entwicklungsumgebungen die genannten Werkzeuge bereits implementieren. Daher muss in diesen Fällen nicht auf eine externe Lösung gesetzt werden. Dies spiegelt aber auch die Akzeptanz und Wichtigkeit dieser Werkzeuge bei der Entwicklung wieder. 
Die Analysen zeigen, dass jedes der vorgestellten intelligenten Werkzeuge den Entwicklungsprozess effizienter gestaltet. Die genannten Gefahren dabei sind meist, dass sich der Entwickler zu sehr auf die Werkzeuge verlässt und diese nicht hinterfragt. Daher sind die Werkzeuge aufgrund ihrer Effizienzsteigerung zu empfehlen, jedoch sollten Entwickler ihnen nicht blind vertrauen und dennoch die Ausgabe selbst kontrollieren. 

Wie aus der Arbeit hervorgeht, sind viele intelligente Werkzeuge jetzt schon in der Softwareentwicklung etabliert. Auch an den Nutzerzahlen der einzelnen Werkzeuge, die in den Marktanalysen genannt wurden, lässt sich dies aufzeigen. Dennoch gibt es auch andere Meinungen. Gerade wie sich in der Marktanalyse zum Thema Refactoring (\autoref{subsec:Refactoring_analyse}) herausgestellt hat, ist das Vertrauen zu gewissen Werkzeugen noch nicht vollständig hergestellt und bedarf noch gewisser Erfahrung. Dennoch betrifft dies nicht alle Werkzeuge. 
Das Themengebiet der vollständig auf künstliche Intelligenz basierten Werkzeuge bietet in Zukunft noch viel Entwicklungspotenzial. Wie zum Beispiel das Werkzeug der Codegenerierung \autoref{sec:codecompletion} zeigt, bietet es immense Möglichkeiten den Entwicklungsprozess zu erleichtern oder gar zu automatisieren. Von einer vollständigen Automatisierung ist hier aber in absehbarer Zeit nicht zu reden. Dennoch entwickelt sich das allgemeine Feld der künstlichen Intelligenz rasant und die Prognosen sind hierzu nur bestätigend \cite{statista.2022}. Deshalb ist davon auszugehen, dass auch die intelligente Werkzeuge immer mehr von dieser Technik profitieren und dadurch weiterentwickelt werden. 
Wie die Arbeit des Weiteren aufzeigt ist das Themengebiet hoch aktuell und sehr wichtig. So werden auch in Zukunft noch einige interessante Neuerungen und Weiterentwicklungen in diesem Bereich stattfinden. 