\section{Kollaboration}
Ein weiteres wichtiges intelligentes Werkzeug, welches sich im Bereich der Dokumentation und Kollaboration befindet, ist die Versionsverwaltung. Diese ermöglicht während des gesamten Projektes die Zusammenarbeit mehrerer Personen, Versionierung, als auch Dokumentation der Änderungen. Große Projekte sind undenkbar ohne Versionsverwaltung.

\subsection{Allgemeines Konzept}
\label{subsubsec:vc}
Die Versionsverwaltung ist gerade bei Kollaborationen mehrerer Entwickler essenziell. Sie wird zum Erfassen von Änderungen an Dateien und Dokumenten verwendet. Eine Versionsverwaltung bietet einige Vorteile. Darunter zählen Transparenz, Übersicht, Nachverfolgbarkeit, Zusammenarbeit mehrerer Entwickler und Identifikation. All dies führt zu einer Effizienzsteigerung. Dabei muss es sich bei zu verwaltenden Daten nicht, wie weitläufig bekannt, nur um Quelltexte handeln. Eine Versionsverwaltung kann auch für Text- oder Tabellendokumente verwendet werden. Im Folgenden wird zwar überwiegend auf die Verwaltung von Quelltexten eingegangen, dennoch soll erwähnt werden, dass eine Versionsverwaltung in den verschiedensten Projektbereichen eingesetzt werden kann. Dies gibt den Entwicklern die Möglichkeit, Änderungen nachzuverfolgen und gegebenenfalls rückgängig zu machen. Des Weiteren lassen sich vielseitige Statistiken über die getätigten Änderungen erstellen. Weitere Hauptaufgaben einer Versionsverwaltung sind Protokollierung, Wiederherstellung, Archivierung, Koordinierung und das gleichzeitige entwickeln mehrere Zweige. 
\begin{comment}
	Ein Hautpelement einer Versionsverwaltung ist das Ermöglichen von continuous integration, continuous delivery (CI/CD). CI/CD ist eine Methode, welche die Automation von verschiedenen Stadien einer Anwendungsentwicklung ermöglicht. 
	
	Ein weitere Möglichkeit die das Werkzeug der Versionsverwaltung durch eine Zentrale Lagerung der Dateien ermöglicht, ist  continuous integration/continuous delivery (CI/CD). 	
\end{comment}
Für die genannten Aufgaben verfügt jede Versionsverwaltung über fünf Basisaktionen, welche diese Aufgaben ermöglichen. 
\begin{itemize}
	\item[(a)] \textbf{Add}: Fügt Dateien zur Versionsverwaltung hinzu.
	\item[(b)] \textbf{Remove}: Entfernt Dateien aus der Versionsverwaltung.
	\item[(c)] \textbf{Commit}: Veröffentlicht vorgenommene Änderungen. 
	\item[(d)] \textbf{Revert}: Setzt die aktuelle Version auf den letzten veröffentlichten Stand zurück. 
	\item[(e)] \textbf{Branch}: Erzeugt einen neuen Zweig aus einem bestehend Zweig. 
	\item[(f)] \textbf{Merge}: Fügt zwei Zweige zusammen und passt Differenzen an. 
\end{itemize}
Darüber hinaus gibt es noch viele weitere Möglichkeiten, die ein Versionsverwaltungssystem besitzen kann. Die genannten Möglichkeiten sind die Grundlagen, die Versionsverwaltungssysteme beherrschen \cite{Davis.2020}. Die Möglichkeiten die dem Entwickler dadurch gegeben werden, sind im Entwicklungsprozess sehr wichtig und werden in den meisten Projekten eingesetzt. Auf Quelltexte bezogen, bietet die zentrale Lagerung der Quelltext-Dateien noch weitere Vorteile. 

Wie in einem Verwaltungssystem üblich, können verschiedene Rollen vergeben werden. Diese ermöglichen die Vergabe von Freigaben und somit eine Rechteverwaltung. Dadurch ist die Möglichkeit gegeben, Änderungen erst nach einer Absprache und/oder Korrektur freizugeben.
Eine weitere Hauptaufgabe, welche Versionsverwaltungssysteme ermöglichen ist die Integration spezieller Abläufe bei der Aktualisierung der Version. So können festgelegte Buildvorgänge, Sicherungen oder verschiedenen Tests der Anwendung durchgeführt werden. 

\subsection{Martkanalyse}
\label{subsubsec:vcanalyze}
Auf dem Markt gibt es unzählige Versionsverwaltungssysteme mit verschiedenen Schwerpunkten für verschiedensten Betriebssysteme. Darunter sind zum Beispiel \textit{Git}, \textit{Subversion (SVN)} und \textit{Concurrent Version System (CVS)}. Das wohl bekannteste Versionsverwaltungssystem ist Git. Git unterscheidet sich insofern von anderen, als das es ein verteiltes System ist. Das bedeutet, dass jede Kopie des Verzeichnisses die gesamten Metadaten inklusive des Änderungsverlaufs enthält und so ein eigenständiges und abgekapseltes Verzeichnis bildet. Git ist kostenlos nutzbar und ein OpenSource-Projekt. Gleiches gilt für Subversion, welches wie git eine große Bekanntheit genießt und von der Apache Software Fundation entwickelt wird. Das Ziel von Subversion ist es, der Nachfolger des in der Vergangenheit sehr bekannten Concurrent Version System zu sein. 

Mit 94 Millionen Nutzern ist GitHub\footnote{https://github.com/} ein sehr bekanntes, auf Git basierendes, Versionsverwaltungssystem im Internet. An diesem Beispiel ist sehr gut zu erkennen, welche weiteren Möglichkeiten eine Versionsverwaltung bietet. Neben Integration der oben genannten Standard-Werkzeugen einer Versionsverwaltung, bietet Github noch viele Erweiterungen rund um die Versionsverwaltung. So können Teams mit Dashboards und verschiedenen Statistiken zum Projekt arbeiten. Auch können verschiedenste Automationen eingepflegt werden, welche den Entwicklungsprozess erleichtern. Bei größerem Interesse kann bei einer Recherche das Stichwort CI/CD  oder DevOps unterstützen. 

\subsection{Nutzen und Risiken}
\label{subsec:documentation_collaboration_risks}
Durch die dadurch VCS entstehenden Möglichkeiten ist eine große Effektivitätssteigerung in Projekten möglich. Sie vereinfachen die Zusammenarbeit in großen Teams und Kollaborationen. Die Werkzeuge sind in den unterschiedlichsten Ausprägungen in vielen Firmen vertreten und werden von vielen Entwicklern schon als Standard angesehen. Zudem sorgen die Möglichkeiten zur Nutzung verschiedener Branches, das Mergen und die Dokumentation der Änderungen generell für eine höhere Sicherheit des Codes. Fehler können schnell identifiziert werden, eine lauffähige Version ist immer vorhanden und es kann unabhängig von anderen entwickelt werden. 

Ein Risiko, welches die Werkzeuge dennoch bieten ist, dass sie über ihre Maße angewandt werden. Auch bei der Versionsverwaltung ist dies ein Thema. Viele Anbieter bieten über die Standardmöglichkeiten einer Versionsverwaltung hinaus Werkzeuge an. Diese sind jedoch nicht immer erforderlich. Ein weiterer Punkt ist, dass die meisten Werkzeuge für den kommerziellen Einsatz in Unternehmen kostenpflichtig sind. Die genannten Werkzeuge werden meist auf Unternehmensinterne Daten angewandt. Deshalb ist auch der Datenschutz ein sehr wichtiger Punkt, welcher nicht vernachlässigt werden sollte. Hierfür bieten die genannten Anbieter zwar Lösungen, wie zum Beispiel die Werkzeuge Unternehmensintern zu hosten, dennoch sollte der Datenschutz nicht vernachlässigt werden. 
