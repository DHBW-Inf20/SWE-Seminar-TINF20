\section{Einleitung}
\label{sec:einleitung}
Das Thema beschäftigt sich mit verschiedenen intelligenten Werkzeugen zur Unterstützung bei der Softwareentwicklung. Viele Prozesse der Softwareentwicklung enthalten repetitive, triviale oder inferentielle Vorgänge. Der Einsatz von intelligenten Werkzeugen soll hier Abhilfe schaffen. Konkrete Vorteile können Zeitersparnis, eine Steigerung der Codequalität oder eine bessere Nachverfolgbarkeit sein. Dies steigert die Wirtschaftlichkeit von Softwareprojekten und schafft dem Entwickler mehr Zeit für komplexe Arbeit. Um die Werkzeuge differenziert zu bewerten ist es notwendig die zugrunde liegenden Konzepte zu verstehen. Somit lassen sich neben den Vorteilen auch vorhandene Grenzen und Risiken erkennen.

Ziel dieser Arbeit ist das Schaffen eines Überblickes über intelligente Werkzeuge zur Erleichterung der Softwareentwicklung. Dabei sollen Konzepte aus den Bereichen Codevervollständigung, Codegenerierung, Codeanalyse, Refactoring, Dokumentation und Kollaboration genauer betrachtet werden. Wesentlicher Bestandteil ist das Herausstellen von Nutzen und Risiken beim Einsatz derartiger Werkzeuge. Zu jedem Konzept wird eine Marktrecherche durchgeführt, welche besonders geeignete Werkzeuge vorstellen soll. Anhand dessen wird eine Handlungsempfehlung gegeben, welche beim Bearbeiten zukünftiger Projekte zur Unterstützung herangezogen werden kann.

Mittlerweile wird die Auswahl an intelligenten Werkzeugen zur Softwareentwicklung immer größer. Neben herkömmlichen Verfahren ermöglicht vor allem künstliche Intelligenz einen großen Fortschritt in diesem Bereich. Aufgrund der Breite dieses Sektors kann nicht jedes intelligente Werkzeug vorgestellt werden. Daher folgt eine Fokussierung auf die bereits aufgezählten Konzepte, da diese sowohl im Unternehmens-, als auch privaten Bereich Potential besitzen. Beispielsweise wären Werkzeuge zum Testen von Software eine weitere Möglichkeit gewesen. Diese sehen wir allerdings aufgrund der Vorlesung \textit{Software Engineering I} als ausreichend bekannt an. Das Thema \textit{Requirements Engineering} bietet auch eine Vielzahl an Werkzeugen an. Es ist allerdings derart umfangreich, dass es wenige Seiten nicht ordnungsgemäß darstellen könnten. Nur eines von vielen Unterthemen wäre das Testmanagement. Außerdem werden derartige Werkzeuge in der Regel vom Unternehmen festgelegt und variieren damit stark. Sie finden im privaten Bereich teilweise kaum Einsatz. Daher werden sie in der folgenden Arbeit nicht näher betrachtet. Der Fokus liegt stattdessen auf Verfahren, welche auch im privaten Bereich, beispielsweise in Freizeitprojekten, Anwendung finden. Informationen zum Thema \textit{Requirements Engineering} können aber beispielsweise unter \cite{chemuturi2013} gefunden werden.