\section{Auswirkungen auf den Entwicklungsprozess}
\label{ch:developmentProzess}

% Nicht nur Gesamtlösungen entwicklen, sondern generische Komponenten, welche man auf mehreren Systemen entwicklen und einsetzen kann
% isoliert an Systemen entwicklen

% TODO: Was ist der Entwicklungsprozess
% - Wleche Phasen umfasst er

Der Entwicklungsprozess beschreibt den Prozess von der Planung über das Design bis zur letztendlichen Implementierung einer Anwendung. Durch die serviceorientierte Architektur gibt es in einigen Bereichen des Entwicklungsprozesses größere oder kleinere Auswirkungen im Gegensatz zu herkömmlichen Architekturen wie zum Beispiel der monolithischen Architektur. Dabei gibt es bei SOA auch andere Schwerpunkte auf welche bei der Entwicklung zu achten sind. Da es nicht nur einen richtigen Weg gibt, eine SOA zu implementieren können die Schwerpunkte je nach spezifischer Implementierung etwas abweichen. 

Der Software-Entwicklungsprozess hat sich über die letzten Jahrzehnte stark weiterentwickelt. Agile Softwareentwicklung wurde immer populärer und damit einhergehend auch die Entwicklung von verteilten Applikationen. Der serviceorientierte Ansatz für die Softwarearchitektur hat sich dabei zu einer wichtigen Alternative gegenüber der traditionellen Softwareentwicklung entwickelt. \cite{Haines.2010}

\subsection{Feldstudie}
\label{sec:feldstudie}

Um die Veränderungen von SOA auf den Entwicklungsprozess genauer zu untersuchen, wurden im Jahr 2010 in einer Feldstudie Softwareentwickler und IT-Manager aus fünf verschiedenen Unternehmen zu diesem Thema befragt. Bei den gestellten Fragen ging es explizit um die Auswirkungen auf die Softwareentwicklung von SOA basierenden Web-Service. Eines der Unternehmen ist ein Consulting Unternehmen und die restlichen vier Unternehmen sind direkt in der Softwareentwicklung tätig. In jedem der Unternehmen wurde SOA bereits etabliert, oder es war zu dem Zeitpunkt der Umfrage dabei zu SOA umzustellen. Die einzelnen Aussagen der Interviewteilnehmer wurden analysiert, gegeneinander verglichen und zusammengefasst. Zur Validierung der Ergebnisse haben die Teilnehmer im Anschluss noch einmal über die Ergebnisse geschaut. Die Resultate wurden in fünf Phasen des Software-Entwicklungsprozesses eingeteilt. Darunter zählen die Planungsphase, Analysephase, Designphase, Implementierungsphase und die Testphase. \cite{Haines.2010}

Im Folgenden werden die aus der Feldstudie ermittelten Unterschiede von SOA zu herkömmlichen Architekturen, wie zum Beispiel dem Monolithen, erläutert. 


\textbf{Planungsphase}

Die erste Phase ist die Planungsphase. In dieser Phase gibt es größere Veränderungen gegenüber der monolithischen Architektur. Die genaue Umgebung der Applikation ist in einer serviceorientierten Architektur in der Planungsphase noch sehr unbestimmt. Viele Aspekte sind mit SOA deutlich schwerer vorherzusagen und zu kontrollieren, vor allem wenn verschiedene Services von unterschiedlichen Bereichen eines Unternehmens oder sogar von anderen externen Unternehmen stammen. Vor allem in der Planungsphase ist es essenziell, dass effektive Kommunikationskanäle zwischen den verschiedenen Stakeholdern aufgebaut werden. Dies ist zwar auch bei anderen Architekturen nötig, doch für SOA ist es wesentlich wichtiger für einen zukünftigen Projekterfolg. Ein weiterer wichtiger Punkt ist es, Standards festzulegen, welche von den Services eingehalten werden müssen. Darunter fallen zum Beispiel Protokolle zur Kommunikation zwischen verschiedenen Services. An diese Standards muss sich bei der Designphase aller benötigten Services gehalten werden. \cite{Haines.2010}


\textbf{Analysephase}

In der Analysephase fallen die wenigsten Veränderungen im Vergleich zu herkömmlichen Architekturen an. Jedoch ist die Analysephase in SOA ein sehr wichtiger Bestandteil des Entwicklungsprozesses. Ein wichtiger Punkt ist dafür zu sorgen, dass alle verwendeten Datenmodelle und Schemata alle benötigten Daten zur Verfügung haben, da bei SOA im Vergleich zu herkömmlichen Architekturen eine globale Perspektive über alle Services hinweg benötigt wird. \cite{Haines.2010}


\textbf{Designphase}

In der Designphase gibt es wie in der Planungsphase wesentliche Unterschiede im Gegensatz zu herkömmlichen Architekturen. Bei SOA ist es sehr wichtig, dass bei der Designphase von Anfang an alle benötigten Schnittstellen definiert werden. Wenn zu einem späteren Zeitpunkt etwas an den Schnittstellen zwischen den Services geändert werden soll, ist mit einem erheblichen Mehraufwand zu rechnen. Wenn bei herkömmlichen Architekturen mit Schnittstellen gearbeitet wird, ist es ebenfalls wichtig diese schon zu Beginn zu definieren, jedoch sind die Auswirkungen bei Änderungen zu einem späteren Zeitpunkt bei vergleichsweise SOA deutlich verstärkt. Die allgemeine Entwicklung von Schnittstellen verändert sich mit SOA auch stark, da in die Services Standards wie zum Beispiel WSDL eingebunden werden müssen. \cite{Haines.2010}

% TODO: - Was ist WSDL?
%       - Service Granularität
%       - Zu viele externe Services = Schlechtes Design



\textbf{Implementierungsphase}

Auf die Implementierungsphase hat SOA die meisten Auswirkungen. Mit SOA können schnell neue und verbesserte Funktionalitäten implementiert werden, ohne mit anderen Geschäftsprozesse dabei in Konflikte zu geraten. Es ist somit einfacher neue Services bereitzustellen und somit hat SOA eine sehr positive Auswirkung auf die Implementierung. Das Verwalten und Management der einzelnen Services ist nun jedoch ein etwas kritischerer Punkt, da jeder Service ein potenzieller \glqq single point of failure\grqq\ einer Anwendung darstellt. Je nach Aufbau der Anwendung kann man dies zum Beispiel mit redundanten Services beschränken. Bei der Implementierung der einzelnen Services ist in der Implementierungsphase ebenfalls ein hohes Maß an Kommunikation und Koordination zwischen den einzelnen, in den Entwicklungsprozess involvierten Gruppen nötig. Durch unterschiedliche Gruppen besteht jedoch auch die Gefahr, dass Services mehrere unterschiedliche Kanäle zur Kommunikation benötigen. Deswegen ist es wichtig die zuvor spezifizierten Standards für SOA einzuhalten. \cite{Haines.2010}


\textbf{Testphase}

Ebenfalls hat SOA einen großen Einfluss auf die Testphase. Dabei muss vor allem mehr Fokus auf die Integrationstests gelegt werden. Bei den Tests sind zwei verschiedene Umgebungen zu beachten. Eine Umgebung, in welcher die Service-Entwickler möglichst einfach Test-Clients erstellen können, um die Services zu testen und eine weitere Umgebung für Client-Entwickler, welche die entwickelten Applikationen gegen Test-Services testen können. Tools für automatisierte Test-Prozesse spielen dabei ebenfalls eine große Rolle. Nicht nur für funktionale Fehler, sondern auch für die Sicherheit und Performance der Services. Durch die Komplexität der Umgebung sind automatisierte Integration-Tests im Gegensatz zu einer monolithischen Architektur wesentlich wichtiger. \cite{Haines.2010}

\textbf{Ergebnis der Feldstudie}

Die Feldstudie zeigt, dass durch SOA der Entwicklungsprozess teilweise stark angepasst werden muss, um ein effektives Arbeiten zu ermöglichen. Einige Bereiche des Entwicklungsprozesses sind dabei stärker betroffen als andere. Ein größeres Augenmerk muss bei SOA auf die Planungs- und Designphase gelegt werden, um die unterschiedlichen Services inklusive deren Schnittstellen korrekt zu definieren. Die Kommunikation zwischen den verschiedenen Teams hat dabei ebenfalls einen besonders großen Stellenwert. Eine weitere Auffälligkeit ist, dass Änderungen an der anfänglichen Planung mit sehr hohen Kosten bzw. Mehraufwand verbunden sind.


\subsection{Agile Softwareentwicklung}
\label{sec:agileDevelopment}

Agile Softwareentwicklung hat in der Vergangenheit immer mehr an Popularität gewonnen. Damit einhergehend hat der serviceorientierte Ansatz für die Entwicklung immer mehr an Bedeutung gewonnen. Größere Unternehmen können nun mit vielen kleineren Entwicklungsteams unabhängig voneinander parallel an einem Projekt produktiv arbeiten. Somit ist zum Beispiel jedes Team für einen Service zuständig. Bei herkömmlichen monolithischen Ansätzen nimmt die Produktivität ab einer gewissen Teamgröße nicht mehr zu oder sogar ab, da sich die Teammitglieder dabei behindern oder in die Quere kommen. Ebenfalls können in einer Applikation unterschiedliche Programmiersprachen für die Services benutzt werden, und somit auf die Anforderungen des Services oder auf die Kompetenzen des jeweiligen Entwicklungsteams angepasst werden. 

Ein wichtiger Punkt, auf welchen in dem Entwicklungsprozess zu achten ist, ist die Service-Granularität. Die Service-Granularität beschreibt den Funktionsumfang eines einzelnen Services. Bei einer hohen Granularität gibt es somit viele Services mit jeweils sehr kleinem Funktionsumfang und bei einer geringen Granularität gibt es wenige Services mit einem größeren Funktionsumfang.\\
In der agilen Softwareentwicklung geht es um die Reduzierung der Größe und des Umfangs der Probleme, der Reduzierung der Zeit für die Implementierung und die Reduzierung der Zeit um Feedback zu enthalten. Dafür bieten sich eine hohe Service-Granularität und somit kleine Services mit sehr beschränkten Funktionalitäten an. Somit können kleine Services mit einem kleinen Funktionsumfang eigenständig entwickelt werden. Wie klein genau ein Service sein sollte, ist jedoch schwer zu pauschalisieren, geschweige denn zu messen. In der Praxis ist jedoch eine höhere Service-Granularität und somit mehrere auf jeweils einen einzelnen Funktionsbereich zugeschnittene Services vorzuziehen. \cite{NADAREISHVILI.2016}

% Service kann einzeln entwicklet, getestestet und bereitgestellt werden, ohne, dass die gesamte Anwendung neu erstellt werden muss.

\subsection{Modularität und Wartbarkeit}
\label{sec:modularity}

Durch die Services ist ebenso ein höheres Level an Modularität in einer Applikation gegeben. Durch die serviceorientierte Architektur muss nur die Kommunikation unter den Services über die Schnittstellen fest definiert sein. Wie ein Service im inneren aufgebaut ist, ist dabei nebensächlich. Verschiedene Services können somit wiederverwendet werden um redundante Softwareentwicklungen vermeiden. Dabei kann nicht nur der Quelltext wiederverwendet werden, sondern teilweise auch ganze Software-Komponenten. Die entwickelten Service-Komponenten können dabei auch in anderen Anwendungen und Systemen eingesetzt werden, um deren Funktionalität zu erweitern. Durch die Wiederverwendbarkeit der Komponenten kann der Entwicklungsprozess langfristig beschleunigt und die Fehleranfälligkeit reduziert werden. Allerdings gibt es dabei andere Schwerpunkte, worauf geachtet werden muss. Bei größeren Änderungen an Services muss darauf geachtet werden, dass die gesamte Applikation mit allen Services noch funktioniert. Gegebenenfalls müssen dabei noch andere Services angepasst werden. Schwerwiegender wird das Problem, wenn der Service in mehreren Applikationen verwendet wird. Dies ist ein weiterer Grund für die Wichtigkeit der Planungsphase bei einer SOA. 

Neben der Modularität bringt auch die Wartbarkeit aus der Sicht des Entwicklungsprozesses langfristig deutliche Vorteile. Durch die Aufteilung in kleine Services können ohne Beachtung von anderen Services, Updates oder Erweiterungen für einen Service implementiert werden. Späteres Refactoring wird dank simplen Services anstelle einer komplexen monolithischen Applikation ebenfalls deutlich vereinfacht. Bei Funktionsupdates können dabei auch weitere Services ohne Probleme implementiert werden. Bei komplexen monolithischen Applikationen ist bei Funktionsupdates mit einem deutlichen Mehraufwand zu rechnen. Dies trifft vor allem zu, wenn einer Applikation über die Zeit immer weitere Funktionalitäten ohne ein größeres Refactoring hinzugefügt werden oder sich viele Altlasten in dieser befinden. \cite{NADAREISHVILI.2016}
