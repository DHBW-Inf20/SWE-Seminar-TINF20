%!TEX root = ../documentation.tex

%% SECTION
\section{Einführung}
\label{sec:einleitung}

Die Entwicklung von großen Software-Projekten kann nur durch Zusammenarbeit gelingen. Dies ist vor allem durch die steigende Komplexität moderner Software bedingt \cite{Wickner:2022:Wie-ist-die-Komplexitat:99}. Um die steigende Komplexität zu beherrschen haben sich eine Vielzahl an Vorgehensmodellen, Strategien und Werkzeugen entwickelt. Welche Unterschiede bestehen dabei zwischen den einzelnen Modellen in Bezug auf die Zusammenarbeit und welche Empfehlungen können für die Wahl ausgesprochen werden? 

%% SUBSECTION
\subsection{Motivation}
\label{sec:einleitung:motivation}

Wie soll der Begriff \emph{groß} im Rahmen dieser Arbeit verstanden werden? Ein \emph{großes} Software-Projekt soll hier insbesondere durch die Anzahl an Entwicklerinnen und Entwickler gekennzeichnet sein. Ein konkreter Vorschlag wäre, ein Software-Projekt \emph{groß} zu nennen, wenn mehr als drei Entwicklerinnen und Entwickler zusammenarbeiten. Dies ist eine recht willkürliche Festlegung, entscheidend soll aber insbesondere die Abgrenzung zur Einzel-Entwicklung sein.
\\
Auch in kleineren und mittleren Unternehmen steigen die Anforderungen an Software, während die Anzahl der Entwicklerinnen und Entwickler meist nicht so schnell gesteigert werden kann. Dadurch fehlen dedizierte Mitarbeiterinnen oder Mitarbeiter für das Management der Zusammenarbeit, während in größeren Unternehmen oft komplette Abteilungen alleine hierfür zuständig sind. In der Folge kommt es zu Problemen in der Zusammenarbeit und daraus resultierend einer geringeren Produktivität oder Fehler in der Software.
\\
Die Zusammenarbeit im Team während eines Projekts ist entscheidend für den Ausgang des Projektes. Dabei spielt die Kommunikation zwischen den Teammitgliedern eine enorme Rolle, vor allem je größer das Projekt und somit die Anzahl der Beteiligten ist \cite{Versteegen:2000:Projektmanagement:25}. Um diese Herausforderung zu bewältigen haben sich verschiedene Modelle entwickelt, auf die im Folgenden kurz eingegangen wird. Außerdem kommt es bei der gemeinsamen Bearbeitung von Quellcode oft zu Konflikten, die aufwändig gelöst werden müssen. Hierfür sollen in dieser Arbeit Strategien vorgestellt werden, um diese Konflikte zu minimieren.



%% SUBSECTION
\subsection{Vorgehensmodelle in der Software-Entwicklung}
\label{sec:einleitung:vorgehensmodelle}
%Hier vielleicht noch Zitate?
Vorgehensmodelle bieten die Möglichkeit ein Rahmenwerk mit festen Strukturen und Prozessen festzulegen, um häufig wiederkehrende Aufgaben bei der Software-Entwicklung zu erleichtern und zu standardisieren. Sie stellen im Grunde einen Plan dar, welcher den Entwicklungsprozess in überschaubare, zeitlich und inhaltlich begrenzte Phasen aufteilt und festlegt, wie die Übergänge zwischen den Phasen zu gestalten sind und welche Werkzeuge oder Werkzeug-Prototypen verwendet werden sollen. Auch die Arten der Zusammenarbeit sind hier mehr oder weniger festgelegt.
\\
Dabei haben sich eine Vielzahl an konkurrierenden Modellen entwickelt, die von ihren jeweiligen Schöpfern angepriesen werden.
Wenn eine grobe Aufteilung vorgenommen werden soll, kann unterschieden werden in die \emph{klassischen} Vorgehensmodelle, die oft aus akademischer oder institutioneller Umgebung stammen und agile Methoden, die sich aufgrund von Problemen mit den bereits vorhandenen Vorgehensmodellen entwickelt haben. Die Klassifizierung von Vorgehensmodellen lässt sich noch weiter verfeinern. Für diese Arbeit reicht aber die übergeordnete Unterteilung und gegebenenfalls wird bei Bedarf direkt darauf eingegangen.

%% SUBSECTION
\subsection{Zielsetzung}
\label{sec:einleitung:zielsetzung}

Diese Arbeit soll einen Überblick über die Zusammenarbeit in verschiedenen Vorgehensmodellen der Software-Entwicklung geben und Unterschiede herausarbeiten. Dabei sollen Vor- und Nachteile für verschiedene Einsatzzwecke aufgezeigt werden. 
Für die Phasen der Implementierung und der Qualitätssicherung soll außerdem eine nähere Betrachtung erfolgen und Werkzeuge und Methoden vorgestellt und bewertet werden, die bei der Lösung der genannten Probleme helfen können.
