%!TEX root = ../documentation.tex

%% SECTION
\section{Zusammenfassung und Ausblick}
\label{sec:fazit}


%\todo{Bisher nur Ideensammlung}

Im Rahmen dieser Arbeit wurde die Bedeutung einer gelungenen Zusammenarbeit bei der Erarbeitung großer Software-Projekte hervorgehoben, wobei  bereits gerade auch sehr kleine Teams große Probleme bei der Zusammenarbeit haben können. Es wurden vier Vorgehensmodelle in Bezug auf die Zusammenarbeit vorgestellt und verglichen. Als Ergebnis lässt sich hier festhalten, dass es auf die gesamte Entwicklungsumgebung und die Anforderungen ankommt, welches Modell geeignet ist. Aufgrund ihrer Komplexität sind Modelle wie das V-Modell oder RUP eher für größere Teams geeignet, wohingegen Scrum bereits von kleinen Teams leicht umgesetzt werden kann. Insgesamt beschäftigt sich die Forschung noch zu wenig mit konkreten Umsetzungen der Zusammenarbeit in verschiedenen Vorgehensmodellen, gerade auch in Bezug auf Werkzeuge.
\\
Die vorgestellten Modelle und Strategien sind dabei lediglich Vorschläge, die in der Regel angepasst werden sollten. Auch wenn diverse Anbieter, gerade auch von Werkzeugen, versprechen, dass diese Out-of-the-Box funktionieren, ist dies selten der Fall. Adam Ruka \cite{Ruka:2022:Big-Tech:06}, Entwickler bei großen Unternehmen wie Amazon und Apple schreibt beispielsweise, dass die großen Tech-Unternehmen oft kein agiles Vorgehensmodell verwenden, das aber auch nicht heißt, dass sie ein Wasserfall-Modell verwenden. Stattdessen haben sie eigene Methoden und Modelle entwickelt, die oft Vorteile aus beiden Welten übernehmen und spezifisch auf die Unternehmenssituation angepasst sind.
Für Unternehmen aus dem Mittelstand ist die Entwicklung eines komplett eigenen Vorgehensmodells oder Strategien für die Zusammenarbeit während der Implementierung aufgrund der Personal-Ressourcen meist nicht so leicht möglich. Hier ist dann die Orientierung an den bekannten Modellen und einer kleineren Anpassung vermutlich der beste Ansatz. Dabei gibt es keine allgemein gültige Lösung. Das richtige Vorgehensmodell sowie die Strategie während der Implementierung hängt von den Anforderungen, der Entwicklungsumgebung und dem Team ab.


