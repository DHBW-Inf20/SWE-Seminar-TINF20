% !TeX encoding = UTF-8
% !TeX program = pdflatex
% !BIB program = biber

%%% Um einen Artikel auf deutsch zu schreiben, genügt es die Klasse ohne
%%% Parameter zu laden.
\documentclass[]{lni}
%%% To write an article in English, please use the option ``english'' in order
%%% to get the correct hyphenation patterns and terms.
%%% \documentclass[english]{class}
%%

\graphicspath{{images/}} %Set graphics path

\newcommand{\citedirect}[1]{\glqq#1\grqq}		% Make the right " around the text for a direct quote.
\newcommand{\citeskip}{$\lbrack$\ldots$\rbrack$}	% Fill in [...] so it won't break up.

% Define a variable which determines whether the width of an included image is greater than the given value. If so, scale, if not take the image width
\makeatletter
\def\maximwidth#1{\ifdim\Gin@nat@width>#1 #1\else\Gin@nat@width\fi}
\makeatother

% Three commands for including images.
\newcommand{\includefigureframe}[3]{%
	\begin{figure}
		\centering
			\frame{\includegraphics[width=\maximwidth{\textwidth}]{#1}}
		#2
		\label{#3}
	\end{figure}%	
}

\newcommand{\includefigurenoframe}[3]{%
	\begin{figure}
		\centering
			\includegraphics[width=\maximwidth{\textwidth}]{#1}
		#2
		\label{#3}
	\end{figure}%	
}

\newcommand{\includefigure}[4][n]{% 
	\IfBeginWith{#1}{f}{%
		\includefigureframe{#2}{#3}{#4}
	}{%
		\includefigurenoframe{#2}{#3}{#4}
	}%
}

% Provides notes and comments while working on the document. 
\usepackage[
	disable,				% Disables all \todo, \missingfigure, \listoftodos in the document. Use for final build so nothing is accidentally left over.
	german,				% Set language of some titles of this package.
	obeyFinal,				% If set in documentclass it will follow.
	backgroundcolor=red,
	linecolor=red,
	textsize=tiny,
	textwidth=3.5cm % Replace with automatic calculated! Right now I don't know how to do this.
]{todonotes}


\usepackage{pgffor}       % Wird benötigt um die \foreach Funktion zu verwenden, um Kapitel dynamisch hinzuzufügen.         

% Bibliography settings
\newcommand{\quoteStyle}{LNI}%{ieee}
\usepackage[
	backend=biber,		% recommended. Alternative: bibtex
	bibwarn=true,
	bibencoding=utf8,	         % If .bib file is encoded with utf8, otherwise ascii
	sortlocale=de_DE,
	style=\quoteStyle,
]{biblatex}                        

\addbibresource{Literatur/literature.bib}  % die Bib-Latex Datei mit Ihrer Literatur                           

\begin{document}
%%% Mehrere Autoren werden durch \and voneinander getrennt.
%%% Die Fußnote enthält die Adresse sowie eine E-Mail-Adresse.
%%% Das optionale Argument (sofern angegeben) wird für die Kopfzeile verwendet.
\title[Zusammenarbeit in der Software-Entwicklung]{Varianten und Herausforderungen der Zusammenarbeit in der Software-Entwicklung - Ein Vergleich verschiedener Herangehensweisen und Werkzeuge}
%%%\subtitle{Untertitel / Subtitle} % if needed
\author[Ingmar Bauckhage \and Vsevolod Pypenko]
{Ingmar Bauckhage\footnote{DHBW Stuttgart Campus Horb, TINF2020, Florianstraße 15, 72160 Horb am Neckar,
Deutschland \email{i20003@hb.dhbw-stuttgart.de}} \and
Vsevolod Pypenko\footnote{DHBW Stuttgart Campus Horb, TINF2020, Florianstraße 15, 72160 Horb am Neckar, Deutschland
\email{i20028@hb.dhbw-stuttgart.de}}}
\startpage{1} % Beginn der Seitenzählung für diesen Beitrag / Start page
\editor{DHBW Stuttgart Campus Horb} % Names of Editors
%\editor{Herausgeber et al.} % Names of Editors
\booktitle{Advanced Software-Engineering 2022} % Name of book title
\yearofpublication{2022}
%%%\lnidoi{18.18420/provided-by-editor-02} % if known
\maketitle

\begin{abstract}
Zusammenarbeit ist bei der Entwicklung von Software-Projekten von großer Bedeutung. Dieser Beitrag vergleicht vier Vorgehensmodelle in Bezug auf die Zusammenarbeit: Wasserfall-Modell, V-Modell, Rational Unified Process und Scrum. Es werden Unterschiede herausgearbeitet sowie Vor- und Nachteile benannt. Die Zusammenarbeit während der Implementierung und des Qualitätsmanagements wird näher betrachtet.\\
Herausforderungen sind z. B. die gemeinsame Arbeit am Quellcode. Hier existieren verschiedene Workflows für die Werkzeuge Git und Github, die veranschaulicht und bewertet werden.\\
Im Qualitätsmanagement ist vor allem die Zusammenarbeit von verschiedenen Teams notwendig. Dies kann über regelmäßigen Austausch mithilfe von formalen Dokumenten und Meetings sichergestellt werden. Darüber hinaus wird beschrieben, welche verschiedenen Möglichkeiten der Verifikation und des Testens es gibt und für welche Projekte diese notwendig sind. 
\end{abstract}
\begin{keywords}
Collaboration \and Zusammenarbeit \and Qualitätsmanagement \and Quellcode-Verwaltung \and Wasserfall-Modell \and V-Modell \and RUP \and Scrum \and Git \and Branching %Keyword1 \and Keyword2
\end{keywords}
%%% Beginn des Artikeltexts
%\section{Überschrift/Heading}
%Bei der immer größer werdenden Komplexität von Softwareprodukten steigen auch die Anforderungen an die Zusammenarbeit. Hilfe bietet dabei die Quellcode-Verwaltung und verschiedene Code-Hosting-Plattformen. Diese Arbeit soll einen Einblick in dieses Themenfeld geben und Vor- und Nachteile aufzeigen.

%%% Alternativ mit Kapiteln dynamisch aus Unterverzeichnis
% Content
\foreach \i in {01,02,03,04,05,06,07,08,09,...,99} {%
	\edef\FileName{content/\i .tex}%
		\IfFileExists{\FileName}{%
			\input{\FileName}
		}
		{%
			% No chapter available
		}
	}

%%% Angabe der .bib-Datei (ohne Endung) / State .bib file (for BibTeX usage)
%\bibliography{mybibfile} %
\printbibliography %if you use biblatex/Biber
\end{document}